% Options for packages loaded elsewhere
\PassOptionsToPackage{unicode}{hyperref}
\PassOptionsToPackage{hyphens}{url}
%
\documentclass[
]{article}
\usepackage{amsmath,amssymb}
\usepackage{lmodern}
\usepackage{ifxetex,ifluatex}
\ifnum 0\ifxetex 1\fi\ifluatex 1\fi=0 % if pdftex
  \usepackage[T1]{fontenc}
  \usepackage[utf8]{inputenc}
  \usepackage{textcomp} % provide euro and other symbols
\else % if luatex or xetex
  \usepackage{unicode-math}
  \defaultfontfeatures{Scale=MatchLowercase}
  \defaultfontfeatures[\rmfamily]{Ligatures=TeX,Scale=1}
\fi
% Use upquote if available, for straight quotes in verbatim environments
\IfFileExists{upquote.sty}{\usepackage{upquote}}{}
\IfFileExists{microtype.sty}{% use microtype if available
  \usepackage[]{microtype}
  \UseMicrotypeSet[protrusion]{basicmath} % disable protrusion for tt fonts
}{}
\makeatletter
\@ifundefined{KOMAClassName}{% if non-KOMA class
  \IfFileExists{parskip.sty}{%
    \usepackage{parskip}
  }{% else
    \setlength{\parindent}{0pt}
    \setlength{\parskip}{6pt plus 2pt minus 1pt}}
}{% if KOMA class
  \KOMAoptions{parskip=half}}
\makeatother
\usepackage{xcolor}
\IfFileExists{xurl.sty}{\usepackage{xurl}}{} % add URL line breaks if available
\IfFileExists{bookmark.sty}{\usepackage{bookmark}}{\usepackage{hyperref}}
\hypersetup{
  pdftitle={Comparing Two Population Means: Bayesian},
  hidelinks,
  pdfcreator={LaTeX via pandoc}}
\urlstyle{same} % disable monospaced font for URLs
\usepackage[margin=1in]{geometry}
\usepackage{color}
\usepackage{fancyvrb}
\newcommand{\VerbBar}{|}
\newcommand{\VERB}{\Verb[commandchars=\\\{\}]}
\DefineVerbatimEnvironment{Highlighting}{Verbatim}{commandchars=\\\{\}}
% Add ',fontsize=\small' for more characters per line
\usepackage{framed}
\definecolor{shadecolor}{RGB}{248,248,248}
\newenvironment{Shaded}{\begin{snugshade}}{\end{snugshade}}
\newcommand{\AlertTok}[1]{\textcolor[rgb]{0.94,0.16,0.16}{#1}}
\newcommand{\AnnotationTok}[1]{\textcolor[rgb]{0.56,0.35,0.01}{\textbf{\textit{#1}}}}
\newcommand{\AttributeTok}[1]{\textcolor[rgb]{0.77,0.63,0.00}{#1}}
\newcommand{\BaseNTok}[1]{\textcolor[rgb]{0.00,0.00,0.81}{#1}}
\newcommand{\BuiltInTok}[1]{#1}
\newcommand{\CharTok}[1]{\textcolor[rgb]{0.31,0.60,0.02}{#1}}
\newcommand{\CommentTok}[1]{\textcolor[rgb]{0.56,0.35,0.01}{\textit{#1}}}
\newcommand{\CommentVarTok}[1]{\textcolor[rgb]{0.56,0.35,0.01}{\textbf{\textit{#1}}}}
\newcommand{\ConstantTok}[1]{\textcolor[rgb]{0.00,0.00,0.00}{#1}}
\newcommand{\ControlFlowTok}[1]{\textcolor[rgb]{0.13,0.29,0.53}{\textbf{#1}}}
\newcommand{\DataTypeTok}[1]{\textcolor[rgb]{0.13,0.29,0.53}{#1}}
\newcommand{\DecValTok}[1]{\textcolor[rgb]{0.00,0.00,0.81}{#1}}
\newcommand{\DocumentationTok}[1]{\textcolor[rgb]{0.56,0.35,0.01}{\textbf{\textit{#1}}}}
\newcommand{\ErrorTok}[1]{\textcolor[rgb]{0.64,0.00,0.00}{\textbf{#1}}}
\newcommand{\ExtensionTok}[1]{#1}
\newcommand{\FloatTok}[1]{\textcolor[rgb]{0.00,0.00,0.81}{#1}}
\newcommand{\FunctionTok}[1]{\textcolor[rgb]{0.00,0.00,0.00}{#1}}
\newcommand{\ImportTok}[1]{#1}
\newcommand{\InformationTok}[1]{\textcolor[rgb]{0.56,0.35,0.01}{\textbf{\textit{#1}}}}
\newcommand{\KeywordTok}[1]{\textcolor[rgb]{0.13,0.29,0.53}{\textbf{#1}}}
\newcommand{\NormalTok}[1]{#1}
\newcommand{\OperatorTok}[1]{\textcolor[rgb]{0.81,0.36,0.00}{\textbf{#1}}}
\newcommand{\OtherTok}[1]{\textcolor[rgb]{0.56,0.35,0.01}{#1}}
\newcommand{\PreprocessorTok}[1]{\textcolor[rgb]{0.56,0.35,0.01}{\textit{#1}}}
\newcommand{\RegionMarkerTok}[1]{#1}
\newcommand{\SpecialCharTok}[1]{\textcolor[rgb]{0.00,0.00,0.00}{#1}}
\newcommand{\SpecialStringTok}[1]{\textcolor[rgb]{0.31,0.60,0.02}{#1}}
\newcommand{\StringTok}[1]{\textcolor[rgb]{0.31,0.60,0.02}{#1}}
\newcommand{\VariableTok}[1]{\textcolor[rgb]{0.00,0.00,0.00}{#1}}
\newcommand{\VerbatimStringTok}[1]{\textcolor[rgb]{0.31,0.60,0.02}{#1}}
\newcommand{\WarningTok}[1]{\textcolor[rgb]{0.56,0.35,0.01}{\textbf{\textit{#1}}}}
\usepackage{graphicx}
\makeatletter
\def\maxwidth{\ifdim\Gin@nat@width>\linewidth\linewidth\else\Gin@nat@width\fi}
\def\maxheight{\ifdim\Gin@nat@height>\textheight\textheight\else\Gin@nat@height\fi}
\makeatother
% Scale images if necessary, so that they will not overflow the page
% margins by default, and it is still possible to overwrite the defaults
% using explicit options in \includegraphics[width, height, ...]{}
\setkeys{Gin}{width=\maxwidth,height=\maxheight,keepaspectratio}
% Set default figure placement to htbp
\makeatletter
\def\fps@figure{htbp}
\makeatother
\setlength{\emergencystretch}{3em} % prevent overfull lines
\providecommand{\tightlist}{%
  \setlength{\itemsep}{0pt}\setlength{\parskip}{0pt}}
\setcounter{secnumdepth}{-\maxdimen} % remove section numbering
\ifluatex
  \usepackage{selnolig}  % disable illegal ligatures
\fi

\title{Comparing Two Population Means: Bayesian}
\author{true}
\date{2021-08-20}

\begin{document}
\maketitle

{
\setcounter{tocdepth}{3}
\tableofcontents
}
\hypertarget{introduction}{%
\section{Introduction}\label{introduction}}

This tutorial will focus on the use of Bayesian (MCMC sampling)
estimation to explore differences between two populations. Bayesian
estimation is supported within R via two main packages
(\texttt{MCMCpack} and \texttt{MCMCglmm}). These packages provide
relatively simple R-like interfaces (particularly \texttt{MCMCpack}) to
Bayesian routines and are therefore reasonably good starting points for
those transitioning between traditional and Bayesian approaches.

\texttt{BUGS} (\emph{Bayesian inference Using Gibbs Sampling}) is an
algorithm and supporting language (resembling R) dedicated to performing
the Gibbs sampling implementation of Markov Chain Monte Carlo method.
Dialects of the \texttt{BUGS} language are implemented within two main
projects;

\begin{itemize}
\item
  \texttt{WinBUGS/OpenBUGS} - written in component pascal and therefore
  originally Windows only. A less platform specific version
  (\texttt{OpenBUGS} - Windows and Linux via some component pascal
  libraries) is now being developed as the WinBUGS successor. Unlike
  WinBUGS, OpenBUGS does not have a Graphical User Interface and was
  designed to be invoked from other applications (such as R). The major
  drawback of WinBUGS/OpenBUGS is that it is relatively slow.
\item
  JAGS (Just Another Gibbs Sampler) - written in C++ and is therefore
  cross-platform and very fast. It can also be called from within R via
  various packages.
\end{itemize}

\begin{Shaded}
\begin{Highlighting}[]
\FunctionTok{library}\NormalTok{(tidyverse)}
\FunctionTok{library}\NormalTok{(coda)}
\FunctionTok{library}\NormalTok{(broom)}
\FunctionTok{library}\NormalTok{(bayesplot)}
\CommentTok{\# Approach specific packages}
\FunctionTok{library}\NormalTok{(MCMCpack)}
\FunctionTok{library}\NormalTok{(R2jags)}
\FunctionTok{library}\NormalTok{(rstan)}
\FunctionTok{library}\NormalTok{(rstanarm)}
\FunctionTok{library}\NormalTok{(brms)}
\end{Highlighting}
\end{Shaded}

\begin{Shaded}
\begin{Highlighting}[]
\CommentTok{\# Assign starter values}
\FunctionTok{set.seed}\NormalTok{(}\DecValTok{1}\NormalTok{)}
\NormalTok{nA }\OtherTok{\textless{}{-}} \DecValTok{60}  \CommentTok{\# sample size from Population A}
\NormalTok{nB }\OtherTok{\textless{}{-}} \DecValTok{40}  \CommentTok{\# sample size from Population B}
\NormalTok{muA }\OtherTok{\textless{}{-}} \DecValTok{105}  \CommentTok{\# population mean of Population A}
\NormalTok{muB }\OtherTok{\textless{}{-}} \FloatTok{77.5}  \CommentTok{\# population mean of Population B}
\NormalTok{sigma }\OtherTok{\textless{}{-}} \DecValTok{3}  \CommentTok{\# standard deviation of both populations (equally varied)}

\CommentTok{\# Data generation}
\NormalTok{yA }\OtherTok{\textless{}{-}} \FunctionTok{rnorm}\NormalTok{(}\AttributeTok{n =}\NormalTok{ nA, }\AttributeTok{mean =}\NormalTok{ muA, }\AttributeTok{sd =}\NormalTok{ sigma)  }\CommentTok{\# Population A sample}
\NormalTok{yB }\OtherTok{\textless{}{-}} \FunctionTok{rnorm}\NormalTok{(}\AttributeTok{n =}\NormalTok{ nB, }\AttributeTok{mean =}\NormalTok{ muB, }\AttributeTok{sd =}\NormalTok{ sigma)  }\CommentTok{\#Population B sample}
\NormalTok{y }\OtherTok{\textless{}{-}} \FunctionTok{c}\NormalTok{(yA, yB)}
\NormalTok{x }\OtherTok{\textless{}{-}} \FunctionTok{factor}\NormalTok{(}\FunctionTok{rep}\NormalTok{(}\FunctionTok{c}\NormalTok{(}\StringTok{"A"}\NormalTok{, }\StringTok{"B"}\NormalTok{), }\FunctionTok{c}\NormalTok{(nA, nB)))  }\CommentTok{\#categorical listing of the populations}
\NormalTok{xn }\OtherTok{\textless{}{-}} \FunctionTok{as.numeric}\NormalTok{(x)  }\CommentTok{\#numerical version of the population category for means parameterization. }
\CommentTok{\# Should not start at 0.}
\NormalTok{data }\OtherTok{\textless{}{-}} \FunctionTok{data.frame}\NormalTok{(y, x, xn)  }\CommentTok{\# dataset}
\FunctionTok{head}\NormalTok{(data)}
\end{Highlighting}
\end{Shaded}

\begin{verbatim}
##          y x xn
## 1 103.1206 A  1
## 2 105.5509 A  1
## 3 102.4931 A  1
## 4 109.7858 A  1
## 5 105.9885 A  1
## 6 102.5386 A  1
\end{verbatim}

\begin{Shaded}
\begin{Highlighting}[]
\FunctionTok{ggplot}\NormalTok{(}\AttributeTok{data =}\NormalTok{ data, }\FunctionTok{aes}\NormalTok{(}\AttributeTok{x =}\NormalTok{ x, }\AttributeTok{y =}\NormalTok{ y)) }\SpecialCharTok{+} 
  \FunctionTok{geom\_boxplot}\NormalTok{() }\SpecialCharTok{+}
  \FunctionTok{theme\_classic}\NormalTok{()}
\end{Highlighting}
\end{Shaded}

\includegraphics{t-test_Bayesian_files/figure-latex/unnamed-chunk-2-1.pdf}

\hypertarget{model-fitting}{%
\section{Model fitting}\label{model-fitting}}

A t-test is essentially just a simple regression model in which the
categorical predictor is represented by a binary variable in which one
level is coded as 0 and the other 1.

For the model itself, the observed response (\(y_{i}\)) are assumed to
be drawn from a normal distribution with a given mean (\(\mu\)) and
standard deviation (\(\sigma\)). The expected values (\(\mu\)) are
themselves determined by the linear predictor (\(\beta_0 + \beta x_i\)).
In this case, β0 represents the mean of the first treatment group and β
represents the difference between the mean of the first group and the
mean of the second group (the effect).

MCMC sampling requires priors on all parameters. We will employ weakly
informative priors. Specifying `uninformative' priors is always a bit of
a balancing act. If the priors are too vague (wide) the MCMC sampler can
wander off into nonscence areas of likelihood rather than concentrate
around areas of highest likelihood (desired when wanting the outcomes to
be largely driven by the data). On the other hand, if the priors are too
strong, they may have an influence on the parameters. In such a simple
model, this balance is very forgiving - it is for more complex models
that prior choice becomes more important.

For this simple model, we will go with zero-centered Gaussian (normal)
priors with relatively large standard deviations (1000) for both the
intercept and the treatment effect and a wide half-cauchy (scale=25) for
the standard deviation.

\[
\begin{align}
y_i &\sim{} N(\mu, \sigma)\\
\mu &= \beta_0 + \beta x_i\\[1em]
\beta_0 &\sim{} N(0,1000)\\
\beta &\sim{} N(0,1000)\\
\sigma &\sim{} cauchy(0,25)\\
\end{align}
\]

\end{document}
